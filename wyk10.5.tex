\documentclass[../main.tex]{subfiles}

\begin{document}

    \subsection{AJAX}
    \begin{itemize}
        \item AJAX (Asynchronous JavaScript and XML) - technika wykonywania zapytań
        HTTP i pobierania z poziomu aplikacji internetowej w przeglądarce WWW bez
        potrzeby przeładowania całej strony.
        \item Bazuje na obiekcie XMLHttpRequest (często używany jest skrót XHR) opisanym
        w standardzie https://xhr.spec.whatwg.org/ .
        \item Implementowany przez wszystkie przeglądarki i dostępny z poziomu kodu
        Javascript.
        \item Umożliwia wysłania żądania HTTP do serwera, pobrania zasobów
        przetworzenia ich w kodzie Javascriptu bez potrzeby przeładowania całej
        strony.
        \item Oryginalnie wykorzystywany przede wszystkim do pobierania danych w
        formacie XML, obecnie bardzo popularny jest JSON.
    \end{itemize}

    Obiekt XMLHttpRequest
    \begin{itemize}
        \item Zwróconą zawartość można uzyskać z pomocą
        \begin{itemize}
            \item responseText - w postaci tekstu
            \item responseXML - jako obiekt dokumentu XML
        \end{itemize}
        \item W nowszych wersjach przeglądarek pole responseType pozwala na
        ustawienie typu zwracanej zwartości. Możliwe są między innymi wartości:
        \begin{itemize}
            \item "text" - wynik zwracay jako string
            \item "document" - wynik zwracany jako dokument HTML (obiekt Document) lub dokument XML (obiekt XMLDocument)
            \item "json" - obiekt Javascript powstały na podstawie przesłanego JSON-a.
        \end{itemize}
        \item Pole readyState może przyjmować następujące wrtaości:
        \begin{itemize}
            \item 0 - zapytanie niezainicjowane
            \item 1 - zapytanie otwarte
            \item 2 - zapytanie wysłane
            \item 3 - odbieranie odpowiedzi
            \item 4 - zapytanie zakończone
        \end{itemize}
        Uwaga: Jeśli przekazywana treść nie jest w odpowiednim formacie otrzymamy
        null.
        \item Metoda setRequestHeader() umożliwia ustawienie nagłówka w wysyłanym
        zapytaniu.
        \item W starych wersjach Internet Explorera (IE 5/6) obiekt XMLHttpRequest należało
        uzyskać poprzez wywołanie
        \item Z powodów bezpieczeństwa współczesne przeglądarki standardowo
        pozwalają jedynie na ładowanie zasobów z tego samego serwera z którgo była
        załadowana strona główna. Jednym ze sposobów na obejście tego problemu
        jest protokół CORS (Cross-domain requests).
    \end{itemize}

    Fetch API - nowy interejs służący do pobierania zasobów.
    \begin{itemize}
        \item Ma zastąpić XMLHttpRequest
        \item Oparte na javasciptowych Promise
        \item API niskopoziomowe!
        \item Mamy kontrolę nad wszystkimi parametrami zapytania HTTP.
        \item Należy samodzielnie ustawiać wszystkie niezbędne nagłówki, ciasteczka, itp
        Jako błąd (Promise.reject()) traktowana jest sytuacja nie możności pobrania
        zasobu w wyniku błędu sieciowego.
        \item Nawet dla odpowiedzi z kodem HTTP 404 obsługa przebiega normalnie, tylko
        status ok jest ustawiany na false.
    \end{itemize}


    \subsection{REST}
    \begin{itemize}
        \item REST = Representational State Transfer
        \item Podejście zaproponowane przez Roya T. Fieldinga w 2000 r. w jego pracy doktorskiej Architectural Styles and the Designs of Network Based Architectures
        \item Oparty na pojęciu zasobu identyfikowanego przy pomocy URI i mogącego posiadać różne reprezentacje (np. XML, JSON).
        \item Operacje na zasobach wykorzystują metody protokołu HTTP takie jak (GET, POST, PUT, DELETE, itd.)
        \item Status operacji są zwracane jako statusy protokołu HTTP
        \item Zakładamy bezstanowość: każda operacja stanowi niezależną całość i serwer może ją zrozumieć i zrealizować bez znajomości poprzednich komunikatów
        \item Format (reprezentacja) zwracanych danych powinna być określana w nagłówku Content-type. Np. text/xml, application/json
        \item W praktyce rzadko spotyka się serwery całkowicie zgodne z wytycznymi REST
        \item Częste odstępstwa:
        \begin{itemize}
            \item Wykorzystanie własnych kodów błędów
            \item Przekazywanie operacji w URL-u zamiast jako metody HTTP (np. POST / delete/book/123)
            \item Przekazywanie formaty zwracanych danych w URL-u jako Content-type (np. GET /book/123?fmt=json)
        \end{itemize}
        \item Czysty REST niezbyt dobrze nadaje się do API zorientowanego na operacje
        zamiast na zasoby.
    \end{itemize}

    \subsection{SOAP}
    \begin{itemize}
        \item SOAP = Simple Object Access Protocol
        \item Starszy i bardziej skomplikowany protokół niż REST
        \item SOAP jest protokołem, a nie jedynie stylem budowania API
        \item Oparty na XML
        \item Można go wykorzystywać do przekazywania danych/zasobów lub do
        wywoływania procedur/operacji
        \item Używany przede wszystkim w ramach protokołu HTTP, ale może działąć na innych protokołach transportowych
        \item Wbudowana obsluga błędów i protokołów bezpieczenistwa (WS-Security:
        m.in. podpisywanie i szyfrowanie wiadomości)
        \item Posiada dodatkowy język opisu udostępnianych operacji i danych (WSDL)
        \item Ma ścisle określoną (dość skomplikowaną) strukturę. Wymaga przekazywania
        sporej ilości dodatkowych danych.
        \item "SOAP is like an envelope, whereas REST is like a postcard"
    \end{itemize}

\end{document}