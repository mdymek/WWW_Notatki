\documentclass[../main.tex]{subfiles}

\begin{document}

    \begin{itemize}
        \item Javascript jest obecna na ok. 95\% stron internetowych
        \item Jest rozwinięciem języka stworzonego przez Brendana Eicha z Netscape (najpierw pod nazwą Mocha, później LiveScript i ostatecznie JavaScript).
        \item W 1996 pojawiła się pierwsza przeglądarka (Netscape Navigator 2.0) zawierającaobsługę Javascriptu
        \item W tym samym roku Microsoft wypuścił Internet Explorera 3.0 z obsługą własnej wersji tego języka pod nazwą JScript.
        \item W 1997 pojawił się standard tego języka pod nazwą ECMAScript* Obecnie Javascript, JScript (Microsoft), ActionScript (Macromedia, a potem Adobe wykorzystany we Flashu i Flexie) są po prostu implementacjami tego standardu (zawierającymi często różne rozszerzenia).
        \item Istnieją kompilatory/translatory (transpiler) tumaczące kod np z ES6 na ES5.
        \item Istnieją rozszerzenia Jascriptu (ECMAScriptu)
        \begin{itemize}
            \item TypeScript - uzywany w Angular i narzęDdziach Microsoft
            \item CoffeeScript
            \item JSX (ReactJs)
        \end{itemize}
    \end{itemize}

    \textbf{IIFE (Immediately Invoked Function Expression)}
    \begin{enumerate}
        \item Funkcja anonimowa zwraca obiekt z funkcjami operującymi na zmiennej 'prywatnej' x
        \item Funkcja anonimowa jest natychmiast wywoływana (z pustą listą argumentów) i wartość zwracana z funkcji jest przypisywana do counter.
    \end{enumerate}
    W taki między innymi sposób są realizowane w JavaScripcie moduły (Module Pattern)


    \textbf{strict mode}
    \begin{itemize}
        \item Włączamy dodając `"use strict"; na początku skryptu lub funkcji
        \item Pojawił się w ECMAScript 5 i jest wspierany w IE from version 10. Firefox from version 4.
        Chrome from version 13. Safari from version 5.1.+ Opera from version 12.
        \item Nawet jeśli nie jest wspierany, można go użyć - nie powoduje to błędów.
        \item W tym trybie wykrywane i zabronione są niektóre potencjalnie niebezpieczne konstrukcje, takie jak np. użycie zmiennej lub obiektu bez wczesniejszej deklaracji.
    \end{itemize}

    \textbf{this}
    \begin{itemize}
        \item Słowo kluczowe this ma trochę inne znaczenie w Javascript niż w C++, czy Javie.
        \item Jego wartość różni się w zależności od kontekstu:
        \begin{itemize}
            \item Na zewnątrz funkcji this oznacza pewiwn obiekt globalny ( w przeglądarkach - window
            \item W 'normalnych' funkcjach w trybie strict - obiekt globalny window, w trybie sloppy - undefined
            \item W konstruktorach i metodach odnosi się do tworzonego/modyfikowanego obiektu.
        \end{itemize}
    \end{itemize}


    \subsection{DOM i BOM}
    Kod w Javascripcie oddziałuje na przeglądarkę i dokumenty w niej wyświetlane przy pomocy interfejsów progratyscznych DOM i BOM.


    \begin{table}[H]
        \begin{center}
            \begin{tabular}{| p{8cm} | p{8cm}| }
                \hline
                \textbf{DOM} - Document Object Model &  \textbf{BOM} - Browser Object Model\\
                \hline
                \hline
                \begin{itemize}
                    \item Odwzorowuje dokument HTML lub XML w postaci drzewa obiektów.
                    \item Wykrywanie elementów: document.getElementById, ByTagName, ByClassName, ByClassName, querySelectorAll(css\_selector).
                    \item Zmiana elementów: element.attribute=, setAttribute(attribute,value), element.innerHTML=, element.style.property=.
                    \item Dodawanie/usuwanie elementów: document.createElement, appendChild, replaceChild, removeChild, write(text).
                    \item Event: onlick, onmouseover, onmouseout
                    \item Najlepiej używać metod: addEventListener, removeEventListener, które pozwalają na przypisanie kilku porcedur obsługi zdarzeń.
                \end{itemize}
                &
                \begin{itemize}
                    \item Opisuje metody i interfejsy umożliwiające interakcję z przeglądarką
                    \item Brak standardu - przeglądarki mogą mieć różniące implementacje
                    \item Główny obiekt - window
                    \item Ważniejsze obiekty i metody:
                    \begin{itemize}
                        \item window.location - URL strony. Umożliwia załadowanie strony z adresu, odświeżenie, itp.
                        \item window.history - Historia przeglądanych stron
                        \item window.screen - Dane dotyczące ekranu
                        \item console - konsola diagnostyczna przeglądarki
                        \item alert, confirm, itd. - Wyświetlanie okien dialogowych
                    \end{itemize}
                \end{itemize}
                \\
                \hline
            \end{tabular}
        \end{center}
    \end{table}


\end{document}