%! suppress = Unicode
%! suppress = MissingImport
%! suppress = SentenceEndWithCapital
%! suppress = LineBreak
%! suppress = MissingLabel
%! suppress = FileNotFound
\documentclass[../main.tex]{subfiles}

\begin{document}
    \subsection{AJAX}
    \begin{itemize}
        \item AJAX (\textbf{Asynchronous JavaScript and XML}) - \textbf{technika wykonywania zapytań}
        HTTP i \textbf{pobierania} z poziomu aplikacji internetowej w przeglądarce WWW \textbf{bez
        potrzeby przeładowania} całej strony.
        \item \textbf{Bazuje} na obiekcie \textbf{XMLHttpRequest} (często używany jest skrót XHR).
        \item Implementowany przez \textbf{wszystkie przeglądarki} i dostępny z poziomu kodu
        Javascript.
        \item Umożliwia \textbf{wysłanie żądania} HTTP do serwera, \textbf{pobranie zasobów},
        \textbf{przetworzenie} ich w kodzie Javascriptu bez potrzeby przeładowania całej
        strony.
        \item Oryginalnie wykorzystywany przede wszystkim do pobierania danych w
        \textbf{formacie XML}, obecnie bardzo popularny jest \textbf{JSON}.
    \end{itemize}

    \subsubsection{Obiekt XMLHttpRequest}
    \begin{itemize}
        \item \textbf{Zwróconą zawartość} można uzyskać z pomocą
        \begin{itemize}
            \item \textbf{responseText} - w postaci tekstu
            \item \textbf{responseXML} - jako obiekt dokumentu XML
        \end{itemize}
        \item W nowszych wersjach przeglądarek pole \textbf{responseType} pozwala na
        \textbf{ustawienie typu} zwracanej zawartości. Możliwe są między innymi wartości:
        \begin{itemize}
            \item \textbf{"text"} - string,
            \item \textbf{"document"} - dokument HTML lub XML,
            \item \textbf{"json"} - obiekt Javascript powstały na podstawie przesłanego JSON-a.
        \end{itemize}
        \item Pole \textbf{readyState} może przyjmować następujące wartości:
        \begin{table}[H]
            \begin{center}
                \begin{tabular}{p{8cm} p{8cm}}
                    \begin{itemize}
                        \item \textbf{0} - zapytanie \textbf{niezainicjowane},
                        \item \textbf{1} - zapytanie \textbf{otwarte},
                        \item \textbf{2} - zapytanie \textbf{wysłane},
                    \end{itemize}
                    &
                    \begin{itemize}
                        \item \textbf{3} - \textbf{odbieranie odpowiedzi},
                        \item \textbf{4} - zapytanie \textbf{zakończone}.
                    \end{itemize}
                \end{tabular}
            \end{center}
        \end{table}
        Uwaga: Jeśli przekazywana treść nie jest w odpowiednim formacie otrzymamy
        \textbf{null}.
        \item Metoda \textbf{setRequestHeader}() umożliwia \textbf{ustawienie nagłówka} w wysyłanym \textbf{zapytaniu}.
        \item Z powodów bezpieczeństwa współczesne przeglądarki standardowo
        pozwalają jedynie na \textbf{ładowanie zasobów z tego samego serwera} z którgo była
        załadowana strona główna. Jednym ze sposobów na obejście tego problemu
        jest protokół \textbf{CORS (Cross-domain requests)}.
    \end{itemize}

    \subsection{Fetch API}
    \begin{itemize}
        \item \textbf{Nowy interfejs} służący do \textbf{pobierania zasobów}, mający zastąpić XMLHttpRequest.
        \item Oparty na \textbf{javasciptowych Promise}. Obiekt Promise reprezentuje \textbf{ewentualne zakończenie} (lub porażkę) \textbf{asynchronicznej operacji} i jej wartości.
        \item API \textbf{niskopoziomowe}.
        \item Mamy \textbf{kontrolę nad wszystkimi parametrami zapytania} HTTP.
        \item Należy \textbf{samodzielnie ustawiać} wszystkie niezbędne \textbf{nagłówki}, ciasteczka, itp
        \item Jako \textbf{błąd} (Promise.reject()) traktowana jest sytuacja niemożności pobrania
        zasobu w wyniku \textbf{błędu sieciowego}.
        \item Nawet dla odpowiedzi z kodem HTTP 404 obsługa przebiega normalnie, tylko
        status ok jest ustawiany na false.
    \end{itemize}

    \subsection{REST}
    \begin{itemize}
        \item REST = \textbf{Representational State Transfer}.
        \item Podejście zaproponowane przez Roya T. Fieldinga w \textbf{2000}r.
        \item Oparty na pojęciu \textbf{zasobu identyfikowanego} przy pomocy \textbf{URI} i mogącego posiadać \textbf{różne reprezentacje} (np. XML, JSON).
        \item Operacje na zasobach wykorzystują \textbf{metody protokołu HTTP} (GET, POST, DELETE, itd.).
        \item \textbf{Statusy operacji} są zwracane jako \textbf{statusy protokołu} HTTP.
        \item Zakładamy \textbf{bezstanowość}: każda operacja stanowi niezależną całość.
        \item \textbf{Format} zwracanych danych powinien być \textbf{określana w nagłówku Content-type}, np. text/xml.
        \item W praktyce \textbf{rzadko spotyka się} serwery \textbf{całkowicie zgodne z wytycznymi} REST.
        \item Częste \textbf{odstępstwa}:
        \begin{itemize}
            \item wykorzystanie \textbf{własnych kodów błędów},
            \item \textbf{przekazywanie} operacji \textbf{w URL-u zamiast} jako \textbf{metody HTTP},
            \item \textbf{przekazywanie formatu} zwracanych danych \textbf{w URL}-u.
        \end{itemize}
        \item Czysty REST \textbf{niezbyt dobrze nadaje się} do API \textbf{zorientowanego na operacje} zamiast na zasoby.
    \end{itemize}

    \subsection{SOAP}
    \begin{itemize}
        \item \textbf{Simple Object Access Protocol}, oparty na \textbf{XML}.
        \item Starszy i \textbf{bardziej skomplikowany} protokół \textbf{niż REST}.
        \item SOAP jest \textbf{protokołem}, a \textbf{nie} jedynie \textbf{stylem budowania} API.
        \item Można go wykorzystywać do \textbf{przekazywania danych}/zasobów lub do \textbf{wywoływania procedur}/operacji.
        \item Używany przede wszystkim \textbf{w ramach protokołu HTTP}, ale \textbf{może działąć na innych} protokołach
        transportowych.
        \item Wbudowana \textbf{obsługa błędów i protokołów bezpieczeństwa} (WS-Security: m.in. podpisywanie i
        szyfrowanie wiadomości).
        \item Posiada \textbf{dodatkowy język opisu udostępnianych} operacji i danych (\textbf{WSDL}).
        \item Ma \textbf{ścisle określoną} (dość skomplikowaną) \textbf{strukturę}. Wymaga przekazywania
        sporej ilości dodatkowych danych.
        \item "SOAP is like an envelope, whereas REST is like a postcard".
    \end{itemize}
\end{document}