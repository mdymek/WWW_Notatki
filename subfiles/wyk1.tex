%! suppress = Unicode
%! suppress = MissingImport
%! suppress = SentenceEndWithCapital
%! suppress = LineBreak
%! suppress = MissingLabel
%! suppress = FileNotFound
\documentclass[../main.tex]{subfiles}

\begin{document}

    \begin{itemize}
        \item \textbf{Frontend}
        \begin{itemize}
            \item \textbf{HTML} - podstawowy \textbf{język znaczników} dla opisu stron www.
            \item \textbf{CSS} (Cascading Style Sheets) - kaskadowe arkusze stylów. Język do definiowania atrybutów, które wpływają na wygląd strony.
            \item \textbf{JavaScript} - de-facto standard dla programowania po stronie klienta. Umożliwia tworzenie interaktywnych stron.
            \item \textbf{jQuery} - jedna z najczęściej stosowanych bibliotek JavaScript do manipulowania elementami strony WWW.
            \item Pozostałe - to bardziej rozbudowane biblioteki lub frameworki do budowy aplikacji WWW.
        \end{itemize}
        \item \textbf{Backend}
        \begin{itemize}
            \item \textbf{Java} - jeden z najpopularnieszych i najbardziej uniwersalnych języków programowania.
            \item \textbf{Serwlety} - niewielkie aplikacje (najczęściej w Javie) rozszerzające możliwości serwera WWW. Odpowiadają na żadania, które serwer otrzymuje od klienta. Najczęściej generują dynamiczne strony WWW lub zwracają dane np. pobrane z bazy danych.
            \item \textbf{Java Server Pages (JSP)} - technologia \textbf{generowania stron} WWW. W specyficznym języku znaczników będącym mieszanką HTML-a i Javy tworzone są szablony stron, które są następnie tłumaczone na serwlety w Javie.
            \item \textbf{Spring} MVC - framework będący częścią Springa pozwalający ułatwiający tworzenie aplikacji webowych bazujących na serwletach w modelu \textbf{Model-View-Controller}.
            \item \textbf{Node.js} - wieloplatformowe \textbf{środowisko uruchomieniowe} (runtime) \textbf{JavaScript} działające po stronie \textbf{serwera}.
            \item \textbf{Django} - \textbf{framework w Pythonie} do szybiego tworzenia aplikacji internetowych bazujących na wzorcu model-template-view.
            \item \textbf{Ruby on Rails} - \textbf{framework w Ruby} do tworzenia aplikacji internetowych.
            \item \textbf{ASP.NET} - \textbf{framework od Microsoftu} do tworzenia \textbf{dynamicznych aplikacji} internetowych przy użyciu m.in. języków C\# i VB.NET.
            \item \textbf{Go} - język stworzony w firmie Google. Doskonale nadaje się do tworzenia aplikacji internetowych (po stronie serwera).
            \item \textbf{Swift} - język stworzony przez Apple dla iOS itp. Na jego bazie istinieją frameworki umożliające tworzenie aplikacji webowych.
        \end{itemize}
    \end{itemize}

    \begin{table}[H]
        \begin{center}
            \begin{tabular}{|p{7.5cm}|p{8.5cm}|}
                \hline
                \multicolumn{2}{|c|}{\textbf{Czym jest aplikacja www?}} \\
                \hline
                \textbf{Klasyczne serwisy WWW} & \textbf{Single Page Applications} \\
                \hline
                \hline
                \begin{itemize}
                    \item Ładowane są \textbf{całe strony}.
                    \item Strony są \textbf{generowane} przez skrypty CGI, serwlety Java, itp.
                    \item \textbf{Zmiana zawartości} strony wymaga załadowania \textbf{całej} strony
                    \item Cała \textbf{logika} aplikacji (lub jej większa część) działa po stronie \textbf{serwera} (backend)
                \end{itemize}
                &
                \begin{itemize}
                    \item Cała strona(y) jest \textbf{ładowana jednorazowo} na początku.
                    \item Cała \textbf{logika} aplikacji działa po stronie \textbf{klienta} (js).
                    \item Różne fragmenty aplikacjji są wyświetlane i/lub tworzone \textbf{dynamicznie} w czasie działania aplikacji.
                    \item Potrzebne dane pobierane z serwera przy pomocy wywołań \textbf{AJAX}.
                \end{itemize} \\
                \hline
            \end{tabular}
        \end{center}
    \end{table}
\end{document}