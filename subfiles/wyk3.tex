\documentclass[../main.tex]{subfiles}

\begin{document}

    \subsubsection{Metody bezpieczne i dempotentne.}

    \begin{theorem}
        \textbf{Metoda bezpieczna} (safe) - metoda, która \textbf{nie generuje efektów ubocznych}. W praktyce są to metody nie modyfikujące danych zasobów na serwerze.
    \end{theorem}

    \begin{theorem}
        \textbf{Metoda idempotentna} - to metoda, której \textbf{wielokrotne wykonanie daje takie same efekty}, jak wykonanie jej jeden raz. Zwrócona przez serwer odpowiedź może się różnić.
    \end{theorem}

    \begin{table}[H]
        \begin{center}
            \begin{tabular}{|p{5cm}|p{5cm}|p{5cm}|}
                \hline
                \textbf{Metoda} & \textbf{Bezpieczna?} & \textbf{Idempotentna?}\\
                \hline
                \hline
                \textbf{GET} & Tak & Tak \\
                \hline
                \textbf{HEAD} & Tak & Tak \\
                \hline
                \textbf{OPTIONS} & Tak & Tak \\
                \hline
                \textbf{TRACE} & Tak & Tak \\
                \hline
                \textbf{PUT} & Nie & Tak \\
                \hline
                \textbf{DELETE} & Nie & Tak \\
                \hline
                \textbf{POST} & Nie & Nie \\
                \hline
            \end{tabular}
        \end{center}
    \end{table}
    Prawidowa implementacja metod leży po stronie twórcy oprogramownia.

    \subsubsection{Nagłówki.}
    \begin{table}[H]
        \begin{center}
            \begin{tabular}{|p{4cm}|p{12cm}|}
                \hline
                \multicolumn{2}{|c|}{\textbf{Ogólne}}\\
                \hline
                \textbf{Connection} & umożliwia tworzenie trwałych połaczeń (od wersji HTTP/1.1). Wysłanie Connection: close wymusza zakończenie połączenia.\\
                \hline
                \hline
                \multicolumn{2}{|c|}{\textbf{Nagłówki żądania}}\\
                \hline
                \textbf{Cookie} & przekazuje zwrotnie do serwera HTTP Cookie (ciasteczko) przesłane wcześniej z serwera w nagłówku Set-Cookie.\\
                \hline
                \textbf{User-Agent} & identyfikuje rodzaj aplikacji klienckiej (przeglądarkę), system operacyjny, klienta, itp.\\
                \hline
                \textbf{Host} & nazwa domenowa serwera. W wersji HTTP/1.1 (i 2.0) nagłowek obowiązkowy. Umożliwia Virtual hosting.\\
                \hline
                \textbf{Accept-Language} & język(i) akceptowane/preferowane przez klienta.\\
                \hline
                \hline
                \multicolumn{2}{|c|}{\textbf{Nagłówki odpowiedzi}}\\
                \hline
                \textbf{Content-Type} & typ MIME zwartości (np. text/html, text/css, image/png).\\
                \hline
                \textbf{Content-Length} & rozmiar ciała (body) odpowiedzi. (W przypadku metody HEAD, rozmiar ciała, które byłoby wysłane dla metody GET).\\
                \hline
                \textbf{Set-Cookie} & HTTP Cookie umożliwia między innymi zapamiętywanie stanu (sesji).\\
                \hline
            \end{tabular}
        \end{center}
    \end{table}
\end{document}