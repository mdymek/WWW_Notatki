%! suppress = Unicode
%! suppress = MissingImport
%! suppress = SentenceEndWithCapital
%! suppress = LineBreak
%! suppress = MissingLabel
%! suppress = FileNotFound
\documentclass[../main.tex]{subfiles}

\begin{document}
    \begin{itemize}
        \item Jest rozwinięciem języka stworzonego przez Brendana Eicha z Netscape (najpierw pod nazwą Mocha, później LiveScript i ostatecznie JavaScript).
        \item W \textbf{1996} pojawiła się pierwsza przeglądarka (Netscape Navigator 2.0) obsługująca Javascript.
        \item W tym samym roku Microsoft wypuścił Internet Explorera 3.0 z obsługą własnej wersji - \textbf{JScript}u.
        \item W \textbf{1997} pojawił się standard tego języka pod nazwą ECMAScript*. Obecne języki są po prostu implementacjami tego standardu (zawierającymi często różne rozszerzenia).
        \item Istnieją \textbf{kompilatory/translatory} (transpiler) tumaczące kod np z ES6 na ES5.
        \item Istnieją \textbf{rozszerzenia} Jascriptu (ECMAScriptu): TypeScript (używany w Angular i narzędziach Microsoftu), CoffeeScript, JSX (ReactJs).
    \end{itemize}

    \subsection{IIFE (Immediately Invoked Function Expression)}
    \begin{enumerate}
        \item Funkcja anonimowa zwraca obiekt z funkcjami operującymi na zmiennej 'prywatnej' x
        \item Funkcja anonimowa jest natychmiast wywoływana (z pustą listą argumentów) i wartość zwracana z funkcji jest przypisywana do counter (która jest obiektem zwracanym jako funkcja).
    \end{enumerate}
    W taki między innymi sposób są realizowane w JavaScripcie moduły (Module Pattern)

    \textbf{strict mode}
    \begin{itemize}
        \item Włączamy dodając "\textbf{use strict}"; na początku skryptu lub funkcji.
        \item Pojawił się w ECMAScript 5 i jest wspierany w IE from version 10. Firefox from version 4.
        Chrome from version 13. Safari from version 5.1.+ Opera from version 12.
        \item Nawet jeśli nie jest wspierany, można go użyć - \textbf{nie powoduje to błędów}.
        \item W tym trybie \textbf{wykrywane i zabronione} są niektóre \textbf{potencjalnie niebezpieczne konstrukcje}, takie jak np. użycie zmiennej lub obiektu bez wczesniejszej deklaracji.
    \end{itemize}

    \textbf{this}
    \begin{itemize}
        \item Słowo kluczowe this ma trochę inne znaczenie w Javascript niż w C++, czy Javie.
        \item Jego wartość \textbf{różni się} w zależności od \textbf{kontekstu}:
        \begin{itemize}
            \item Na zewnątrz funkcji this oznacza pewiwn obiekt globalny (w przeglądarkach - window).
            \item W "normalnych" funkcjach w trybie \textbf{strict} - obiekt globalny \textbf{window}, w trybie \textbf{sloppy} - \textbf{undefined}.
            \item W \textbf{konstruktorach} i \textbf{metodach} odnosi się do tworzonego/modyfikowanego \textbf{obiektu}.
        \end{itemize}
    \end{itemize}


    \subsection{DOM i BOM}
    Kod w Javascripcie oddziałuje na przeglądarkę i dokumenty w niej wyświetlane przy pomocy interfejsów DOM i BOM.


    \begin{table}[H]
        \begin{center}
            \begin{tabular}{| p{8cm} | p{8cm}| }
                \hline
                \textbf{DOM} - Document Object Model &  \textbf{BOM} - Browser Object Model\\
                \hline
                \hline
                \begin{itemize}
                    \item Odwzorowuje dokument HTML lub XML w postaci \textbf{drzewa obiektów}.
                    \item Wykrywanie elementów: \textbf{document} .getElementById, .ByTagName, .ByClassName, .querySelectorAll(css\_selector).
                    \item Zmiana elementów: \textbf{element} .attribute=, .setAttribute(attribute,value), .innerHTML=, .style.property=.
                    \item Dodawanie/usuwanie elementów: document.createElement, appendChild, replaceChild, removeChild, write(text).
                    \item \textbf{Event}: onlick, onmouseover, onmouseout
                    \item Najlepiej używać metod: \textbf{addEventListener}, removeEventListener, które pozwalają na \textbf{przypisanie kilku procedur} obsługi zdarzeń.
                \end{itemize}
                &
                \begin{itemize}
                    \item \textbf{Opisuje} metody i interfejsy umożliwiające \textbf{interakcję z przeglądarką}.
                    \item Brak standardu - przeglądarki mogą mieć różniące się implementacje.
                    \item Główny obiekt - \textbf{window}.
                    \item Ważniejsze obiekty i metody:
                    \begin{itemize}
                        \item window.\textbf{location} - URL strony. Umożliwia załadowanie strony z adresu, odświeżenie, itp.,
                        \item window.\textbf{history} - historia przeglądanych stron,
                        \item window.\textbf{screen} - dane dotyczące ekranu,
                        \item \textbf{console} - konsola diagnostyczna przeglądarki,
                        \item \textbf{alert, confirm} itd. - wyświetlanie okien dialogowych.
                    \end{itemize}
                \end{itemize}\\
                \hline
            \end{tabular}
        \end{center}
    \end{table}
\end{document}