%! suppress = Unicode
%! suppress = MissingImport
%! suppress = SentenceEndWithCapital
%! suppress = LineBreak
%! suppress = MissingLabel
%! suppress = FileNotFound
\documentclass[../main.tex]{subfiles}

\begin{document}
    \subsection{OWASP Top 10 Application Security Risks 2017}
    \textbf{Open Web Application Security Project} - organizacja non-profit zajmująca się \textbf{zagrożeniami i zabezpieczaniami} aplikacji
    webowych.
    \begin{table}[H]
        \begin{center}
            \begin{tabular}{p{7cm} p{10cm}}
                \begin{itemize}
                    \item A1:2017 - Injection
                    \item A2:2017 - Broken Authentication
                    \item A3:2017 - Sensitive Data Exposure
                    \item A4:2017 - XML External Entities (XXE)
                    \item A5:2017 - Broken Access Control
                \end{itemize}
                &
                \begin{itemize}
                    \item A6:2017 - Security Misconfiguration
                    \item A7:2017 - Cross-Site Scripting (XSS)
                    \item A8:2017 - Insecure Deserialization
                    \item A9:2017 - Using Components with Known Vulnerabilities
                    \item A10:2017 - Insufficient Logging \& Monitoring
                \end{itemize}
            \end{tabular}
        \end{center}
    \end{table}

    \subsection{SQL Injection}
    \begin{itemize}
        \item Polega na \textbf{niedostatecznej walidacji} i \textbf{filtrowaniu} wprowadzonych przez
        użytkownika danych wykorzystywanych do tworzenia \textbf{zapytania SQL}.
        \item Dane te są \textbf{interpretowane} przez bazę \textbf{jako elementy języka SQL}, a nie
        zwykłe dane tekstowe.
        \item \textbf{Zabezpieczenia}:
        \begin{itemize}
            \item Używanie \textbf{prepared (parametrized) query} - szkielet tych zapytań jest \textbf{wstępnie parsowany}/kompilowany na serwerze
            bazy danych, a dopiero później uzupełniany o parametry.
            \item \textbf{Czyszczenie} (sanitizing) wprowadzonych \textbf{danych z niebezpiecznych znaków}
            (apostrofy, cudzysłowy, itp.). Znaki te można usuwać, albo \textbf{zastępować bezpiecznymi kombinacjami} (\textbf{escaping}).
        \end{itemize}
    \end{itemize}

    \subsection{Cross-Site Scripting (XSS)}
    \begin{itemize}
        \item \textbf{W treści} atakowanej \textbf{strony wstawiany jest kod} (najczęściej Javascript)
        wykonujący \textbf{niepożądaną akcję}.
        \item Kod ten jest \textbf{wykonywany} jest z \textbf{aktualnymi uprawnieniami} użytkownika, co
        umożliwia mu dostęp i ewentualną zmianę danych prywatnych użytkownika
        \item Zabezpieczenia:
        \begin{itemize}
            \item Podobnie jak w przypadku SQl Injection, \textbf{sprawdzanie odbieranych przez
            użytkownika danych}. Usuwanie lub zamiana na zwykły tekst wszystkich
            instrukcji HTML. javascript, itp.
            \item Używanie \textbf{cookies} z flagą \textbf{HttpOnly} (oraz \textbf{Domain}) co zabezpiecza przed
            dostępem do ciasteczek z poziomu skryptów. Skrypt nie może odczytać
            ciasteczka i przesłać do atakującego.
            \textbf{Nie zabezpiecza} nas to jednak przed wykonaniem złośliwej akcji \textbf{w
            ramach istniejącej sesji} na serwerze na który użytkownik jest zalogowany.
            Przeglądarka automatycznie dołącza ciasteczka do zapytań idących do
            serwera z pasującym Domain.
        \end{itemize}
    \end{itemize}

    \subsection{Session hijacking}
    \begin{itemize}
        \item Atakujący \textbf{przechwytuje identyfikator} sesji lub jest go w stanie \textbf{przewidzieć} i
        \textbf{wygenerować}.
        \item Posiadając identyfikator może \textbf{połączyć się} z serwerem \textbf{w ramach sesji
        autoryzowanej} już przez jej właściciela i wykonać złośliwe działania.
        \item Fatalnym pomysłem jest np. nadawanie kolejnych numerów jako identyfikatory sesji!
        \item Zabezpieczenia:
        \begin{itemize}
            \item \textbf{generowanie niemożliwych do przewidzenia} identyfikatorów sesji,
            \item używanie \textbf{HTTPS},
            \item \textbf{umożliwienie} użytkownikowi \textbf{wylogowania się} powodujące dezaktywacja sesji
            i jej identyfikatora.
        \end{itemize}
    \end{itemize}

    \subsection{Session Fixation}
    \begin{itemize}
        \item Wiele serwisów przy pierwszym połączeniu użytkownika do serwisu
        rozpoczyna sesję (nadaje użytkownikowi session ID), a następnie, już w
        ramach tej sesji, pozwala mu się zalogować do serwisu.
        \item W takiej sytuacji czasami możliwy jest atak Session Fixation:
        \begin{enumerate}
            \item Atakujący łączy się do strony logowania.
            \item Serwer zawraca stronę logowania z identyfikatorem sesji.
            \item \textbf{Atakujący przesyła ofierze link do strony logowania z ustawionym swoim identyfikatorem sesji}.
            \item Ofiara loguje się na serwer używając przekazanego identyfikatora sesji.
            \item Użytkownik ładuje stronę z kontem ofiary używając \textbf{wspólnego identyfikatora
            sesji}, która została już wcześniej autoryzowana przez ofiarę.
        \end{enumerate}
        \item \textbf{Zabezpieczenia}:
        \begin{itemize}
            \item \textbf{Zablokowanie} przyjmowania Session ID \textbf{w URL}-u i \textbf{parametrach POST}.
            Zabezpiecza przed pierwszymi dwoma typami ataków.
            \item Generowanie nowej sesji (\textbf{nowy Session ID}) po zalogowaniu użytkownika.
        \end{itemize}
    \end{itemize}

    \subsection{Cross-Site Request Forgery}
    \begin{itemize}
        \item Definicja wg. OWASP: A CSRF attack forces a logged-on victim's browser to send a
        forged HTTP request, including the victim's session cookie and
        any other automatically included authentication information, to a
        vulnerable web application. This allows the attacker to force the
        victim's browser to generate requests the vulnerable application
        thinks are legitimate requests from the victim.
        \item Przykładowy scenariusz:
        \begin{enumerate}
            \item Atakujący chce wykonać nieautoryzowany przelew z banku, który udostępnia
            taką funkcjonalność przez wywołania postaci.
            \item Ofiara jest zalogowana na serwer (np. do banku).
            \item Atakujący podsuwa ofierze stronę, maila, itp. zawierającą link wykonujący
            złośliwą operację na serwerze (np. przelew bankowy).
            \item Ofiara \textbf{uruchamia} podsunięty \textbf{link w przeglądarce}, w której \textbf{aktywna jest sesja}
            na serwer. W przypadku drugiego linka, ofiara nie musi nawet uruchamiać
            linka - zostanie on automatycznie uruchomiony w celu pobrania "obrazka".
            \item Link zawierający żądanie dla serwera zostaje przesłany na serwer wraz ze
            wszystkimi ciasteczkami (w tym z identyfikatorem sesji).
        \end{enumerate}
        \item \textbf{Zabezpieczenia}:
        \begin{itemize}
            \item \textbf{Synchronizer Tokens} - Każda \textbf{operacja zmieniająca stan} aplikacji \textbf{wymaga dodatkowego} unikalnego \textbf{jednorazowego tokenu} (CSRF token).
            \item \textbf{Dodatkowa autoryzacja} przez użytkownika - dodatkowe hasło, CAPTCHA, token jednorazowy.
            \item Atrybut \textbf{SameSite} w ciasteczku - zapobiega wysłaniu ciasteczka jeśli \textbf{zapytanie nie
            pochodzi z tej samej strony}/ serwisu, z której pochodzi ciasteczko.
            \item \textbf{CORS}.
        \end{itemize}
    \end{itemize}

    \subsection{SOP i CORS}

    \subsubsection{Same Origin Policy (SOP)}
    \begin{itemize}
        \item Aby zabezpieczyć się przed CSRF zasób (skrypt) pobrany z danej lokalizacji
        (Origin) może czytać dane i wysyłać jedynie do innych  \textbf{zasobów z tej samej
        lokalizacji}).
        \item Lokalizacja = scheme + host + port.
        \item \textbf{Problemy z SOP} - czasami chcemy zezwolić na dostęp do naszych zasobów z zewnątrz. Na
        przykład udostępniając  \textbf{publiczne API}, fonty, skrypty.
    \end{itemize}

    \subsubsection{Cross-Origin Resource Sharing (CORS)}
    \begin{itemize}
        \item \textbf{Każde żądanie} w trybie CORS powoduje dodanie prze przeglądarkę do
        zapytania nagłówka \textbf{Origin} z adresem serwisu z którego pochodzi strona
        wysyłająca żądanie.
        \item \textbf{Nagłówek} jest \textbf{ustawiany przez przeglądarkę} i kod użytkownika nie ma
        możliwości zmiany tego nagłówka.
        \item Serwer odpowiada z nagłówkiem.
        \item Przeglądarka sprawdza, czy zwrócona wartość pozwala na udostępnienie
        zasobu i przekazuje go do kodu strony tylko wtedy, gdy jest to dozwolone.
        \item Istnieje tez możliwość \textbf{wymuszenia dodatkowej autoryzacji}.
    \end{itemize}
\end{document}