\documentclass[a4paper]{article}

\usepackage{fullpage} % Package to use full page
\usepackage{parskip} % Package to tweak paragraph skipping
\usepackage{tikz} % Package for drawing
\usepackage{amsmath}
\usepackage{hyperref}
\usepackage[utf8]{inputenc}
\usepackage{lmodern}
\usepackage[MeX]{polski}
\usepackage[T1]{fontenc}
\usepackage{graphicx}
\usepackage{float}
\usepackage{subfiles}
\usepackage{booktabs}
\graphicspath{{graphics/}}
\usepackage{multirow}
\usepackage{xparse}
\usepackage[most]{tcolorbox}
\usepackage{fancyvrb,newverbs,xcolor}

\definecolor{cverbbg}{gray}{0.93}

\definecolor{battleshipgray}{rgb}{0.52, 0.52, 0.51}

\newenvironment{lcverbatim}
{\SaveVerbatim{cverb}}
{\endSaveVerbatim
\flushleft\fboxrule=0pt\fboxsep=.5em
\colorbox{cverbbg}{%
\makebox[\dimexpr\linewidth-2\fboxsep][l]{\BUseVerbatim{cverb}}%
}
\endflushleft
}

\NewDocumentCommand{\newframedtheorem}{O{}momo}{%
\IfNoValueTF{#3}
{%
\IfNoValueTF{#5}
{\newtheorem{#2}{#4}}
{\newtheorem{#2}{#4}[#5]}%
}
{\newtheorem{#2}[#3]{#4}}
\tcolorboxenvironment{#2}{#1}%
}

\newframedtheorem{theorem}{Theorem}[section]
\newframedtheorem[
enhanced jigsaw,
colback={white!40!yellow},
colframe=red,
boxrule=2pt,
sharp corners,
]{definition}[theorem]{Definition}

\title{Notatki z kursu Programowanie dla WWW}
\author{Małgorzata Dymek}
\date{2019/20, semestr zimowy}

\begin{document}
    \maketitle

    \section{Wstęp.}
    \subfile{wyk1}

    \section{URL i HTTP.}
    \subfile{wyk2}
    \subfile{wyk3}

    \section{HTML - HyperText Markup Language.}
    \subfile{wyk4}

    \section{CSS.}
    \subfile{wyk5}

    \section{Javascript.}
    \subfile{wyk6}

    \section{React.}
    \subfile{wyk7}

    \section{Node.js.}
    \subfile{wyk8}

    \section{Express.js.}
    \subfile{wyk9}

    \section{Zarządzanie sesjami w aplikacjach webowych.}
    \subfile{wyk10}

    \section{Komunikacja Frontend-Backend.}
    \subfile{wyk10.5}

    \section{Spring vs JEE.}
    \subfile{wyk11}

    \section{Bezpieczeństwo.}
    \subfile{wyk12}
\end{document}