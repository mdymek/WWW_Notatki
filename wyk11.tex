\documentclass[../main.tex]{subfiles}

\begin{document}
    \subsection{JEE}
    \begin{itemize}
        \item \textbf{Java for Enterprise Edition}
        \item Platforma do \textbf{tworzenia aplikacji biznesowych}
        \item Jest \textbf{zestawem specyfikacji}, a \textbf{nie implementacji}.
        \item Twórcy specyfikacji (Sun Microsystems, a następnie Oracle) udostępniają również \textbf{wzorcową implementację} specyfikcaji JEE (obecnie: GlassFish lub Sun Java System Application Server)
        \item Podstawowe \textbf{interfejsy} programistyczne zdefiniowane w pierwszych wersjach, to
        \begin{itemize}
            \item \textbf{JDBC} (Java Database Conectivity) - dostęp do baz danych
            \item \textbf{Java Servlets} - obsługa komunikacja sieciowej (najczęściej HTTP)
            \item \textbf{JSP} (Java Server Pages) - dynamiczne strony WWW
            \item \textbf{EJB} (Enterprise Java Beans) - specyfikacja komponentów biznesowych po stronie serwera
        \end{itemize}
        \item Niektóre z tych API zostały przeniesione do "zwykłej" Javy (JavaSE - Java Standard Edition), a JEE wzbogaciła się o wiele innych interfejsów
        \item \textbf{Problemy z JEE}:
        \begin{itemize}
            \item Pierwsze wersje miały dużo możliwości, ale były \textbf{skomplikowane} w użyciu (szczególnie EJB)
            \item Aplikacje JEE wymagały \textbf{dużego} (ciężkiego) \textbf{serwera} implementującego pełen zakres specyfikacji
            \item W związku z tym pojawiły się konkurencyjne/komplementarne rozwiązania: \textbf{Spring}, Struts, Guiceł
        \end{itemize}
    \end{itemize}

    \textbf{Serwery aplikacji} application servers) - \textbf{implementują pełną specyfikację} JEE między innymi z EJB (Java beans) i
    JMS (Java messaging).

    \subsection{Spring}
    Rozwiązuje pewne problemy z JEE:
    \begin{itemize}
        \item \textbf{Upraszcza tworzenie aplikacji biznesowych}, szczególnie tych, które nie
        potrzebowały dużej części interfejsów JEE
        \item Pozwala na pracę ze \textbf{zwykłymi obiektami Javy} (\textbf{POJO - Plain Old Java
        Object} zamiast skomplikowanych EJB).
        \item Jest \textbf{zestawem bibliotek} zawierących implementacje, co umożliwia
        wdrażanie aplikacji Springa na serwerach WWW nie implementujących
        JEE, a nawet budowanie aplikacji desktopowych.
        \item Opiera na \textbf{IoC (Inversion of Control)} i \textbf{Dependency Injection}, co
        pozwala między innymi ograniczyć stosowanie niezbyt wygodnych usług
        nazewnicznych JNDI (Java Naming and Directory Interface)
        \item Różne \textbf{sposoby konfiguracji} systemu:
        \begin{itemize}
            \item \textbf{pliki XML}
            \item \textbf{adnotacje}
            \item \textbf{kod Javy}
        \end{itemize}
        \item \textbf{Dwa warianty budowy} aplikacji webowych: standardowe \textbf{serwlety} oraz nieblokujące technologie \textbf{Reactive stack}.
        \item Moduły Springa można wykorzystać także do budowy \textbf{aplikacji desktopowych}
        \item Można budować aplikacje zawierające \textbf{wbudowany serwer WWW}.
        \item Spring Boot umożliwiający \textbf{łatwe tworzenie różnych typów aplikacji} bazujący
        na domyślnych sensowych ustawieniach
    \end{itemize}

    \subsubsection{Kontener IoC}
    \begin{itemize}
        \item Filozofia Spring-a opiera się na \textbf{odwróceniu zależności} (IoC) i \textbf{wstrzykiwaniu
        zależności} (DE).
        \item Najważniejszą częscią architektury Springa jest kontener IoC w
        terminologii Spring-a zwany \textbf{kontekstem aplikacji}, implementujący interfejs
        ApplicationContext.
        \item Kontekst aplikacji \textbf{zarządza cyklem życia} komponentów (beans) tworzących
        aplikację.
        \item Komponenty to \textbf{zwykłe obiekty Javy} (POJO).
    \end{itemize}

    \subsubsection{Wstrzykiwanie zależności w Springu}
    \begin{itemize}
        \item przy pomocy \textbf{XML}-a
        \item przy pomocy \textbf{adnotacji}
        \item wstrzykiwanie zależności przy pomocy:
        \begin{itemize}
            \item \textbf{konstruktora} - zalecane dla wymaganych zalezności
            \item metod ustawiających (\textbf{setterów}) - najlepiej dla zależności opcjonalnych
            \item \textbf{pól w klasach}
        \end{itemize}
    \end{itemize}

    \subsubsection{Bean scopes}
    \begin{table}[H]
        \begin{center}
            \begin{tabular}{| p{3.5cm} | p{12.5cm} |}
                \hline
                \multicolumn{2}{|c|}{\textbf{Ogólne zakresy.}}\\
                \hline
                \textbf{singleton (default)} &
                tworzona tylko \textbf{jedna instancja} (w jednym ApplicattionContext)\\
                \hline
                \textbf{prototype} & Z każdym \textbf{żądaniem} pobrania bean'a zwracany jest
                \textbf{nowy egzemplarz}\\
                \hline
                \hline
                \multicolumn{2}{|c|}{\textbf{Zakresy dostępne ApplicationContext typu webowego}}\\
                \hline
                \textbf{request} & \textbf{Pojedyncza instancja} jest tworzona i dostępna podczas
                \textbf{trwania} jednego \textbf{żądania} HTTP (HTTP request)\\
                \hline
                \textbf{session} &\textbf{Pojedyncza instancja} jest tworzona i dostępna podczas
                trwania jednej \textbf{sesji} HTTP (HTTP session)\\
                \hline
                \textbf{application} & \textbf{Pojedyncza instancja} jest tworzona i dostępna podczas
                trwania \textbf{ServletContext}.\\
                \hline
                \textbf{websocket} & \textbf{Pojedyncza instancja} jest tworzona i dostępna podczas
                trwania \textbf{WebSocket}.\\
                \hline
            \end{tabular}
        \end{center}
    \end{table}

    \subsubsection{MVC}
    \begin{itemize}
        \item \textbf{Model} - dane aplikacji (najczęściej POJO)
        \item \textbf{View} - prezentuje dane z modelu dla użytkownika (w naszym wypadku generuje
        stronę HTML)
        \item \textbf{Controller} - przetwarza żądania użytkownika generując/modyfikując model i przekazuje
        go do widoku w celu wyświetlenia
    \end{itemize}

    \subsubsection{Dispatcher Servlet} - w Spring MVC główym \textbf{obiektem synchronizującym} te \textbf{aktywności} jest
    DispatcherServlet
    \begin{itemize}
        \item \textbf{Handler mapping} - \textbf{mapuje żądania} HTTP \textbf{na} odpowiednie \textbf{kontrolery}
        \item \textbf{Controller} - \textbf{wywołuje odpowiednie metody} obsługujące żądania GET, POST, itp. Metody te
        \textbf{ustawiają odpowiedni model} danych i zwracają nazwę widoku.
        \item \textbf{View Resolver} - \textbf{mapuje nazwy widoków} na odpowiednie \textbf{szablony stron} (np. JSP)
        \item \textbf{View} - \textbf{Renderuje stronę} na podstawie szablonu z odpowiednim modelem
    \end{itemize}
\end{document}