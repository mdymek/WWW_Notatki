\documentclass[../main.tex]{subfiles}

\begin{document}
    \subsection{XHTML}
    Znacznie bardziej restrykcyjny, bo \textbf{zgodny z XML}.
    \begin{itemize}
        \item Znaczniki i atrybuty w muszą być pisane \textbf{małymi literami}.
        \item Wszystkie elementy muszą być \textbf{poprawnie zakończone} (np. <br/>).
        \item Wszystkie wartości atrybutów muszą być w \textbf{cudzysłowiach} lub \textbf{apostrofach}.
        \item Każdy atrybut musi mieć \textbf{wartość}.
        \item Bardziej \textbf{surowe zasady zagnieżdżania} (np. zakazane a w a, form w form itd).
        \item Znaki $< >$ \& ' " muszą być reprezentowane przez \textbf{encje} (np. \&amp) nawet w wartościach atrybutów.
        \item Dla znaczników a, form, map itd należy używać \textbf{id zamiast name}.
        \item Dokument musi zaczynać się od \textbf{deklaracji} <!DOCTYPE html PUBLIC \dots XHTML \dots
        \item Dokument musi zawierać elementy <html>, <head>, <title> i <body>.
    \end{itemize}

    \subsection{HTML 5}

    \begin{table}[H]
        \begin{center}
            \begin{tabular}{p{8cm} p{8cm}}
                \begin{itemize}
                    \item Dużo \textbf{wymogów} dotyczących \textbf{poprawności dokumentów} i ich dokładniejsze sprawdzanie.
                    \item Możliwość zastosowania \textbf{narzędzi pochodzących z xml}
                    \item Wsparcie \textbf{przestrzeni nazw}
                    \item Zintegrowane \textbf{wsparcie dla multimediow} niewymagające użycia wtyczek.
                    \item \textbf{Brak wersji} (Strict, Frames, Transitional).
                    \item Dobrze tolerowany przez stare przeglądarki.
                    \item \textbf{Nowe elementy}, np.: article, aside, audio, footer, header.

                    \item \textbf{Nowe API}
                    \begin{itemize}
                        \item \textbf{GeoLocation API} - lokalizacja użytkownika,
                        \item \textbf{Web Storage API} - wygodniejsze składowanie informacji po stronie klienta,
                        \item \textbf{Web Socket API} - full-duplex komunikacja z serwerem,
                        \item \textbf{Web Worker API} - umożliwia uruachamianie zdaań w Javascripcie w innym wątku niż interfejs przeglądarki,
                        \item \textbf{Drag \& Drop API} - wsparcie dla "drag and drop".
                    \end{itemize}
                \end{itemize}
                &
                \begin{itemize}
                    \item \textbf{Nowe atrybuty}: charset (dla elementu meta), async (dla elementu script).
                    \item \textbf{Globalne atrybuty} (które mogą być zastosowane do dowolnego elementu): id, tabindex, hidden, data-* (custom data attributes).
                    \item Usunięto niektóre \textbf{zdeprecjonowane elementy}: acronym, applet, basefont, big, center, dir, font, frame, frameset, isindex, noframes, strike.
                    \item \textbf{Zmiana znaczenia}/użycia niektórych istniejących atrybutów i elementów (np <i> i <small> - zmiana znaczenia na bardziej semantyczne).
                    \item \textbf{Nowe typy elementów formularzy}: dates and times, email, url, search, range, tel, color.
                    \item \textbf{Content type}
                    \begin{itemize}
                        \item Dla zawartości w standarcie HTML: text/html.
                        \item Dla zawartości w standarcie XHTML: application/xhtml+xml (ewentualnie application/xml lub text/xml).
                    \end{itemize}
                \end{itemize}
            \end{tabular}
        \end{center}
    \end{table}

    \textbf{Interpretacja przez przeglądarkę.}
    \begin{table}[H]
        \begin{center}
            \begin{tabular}{|p{4cm}|p{4cm}|p{4cm}|p{4cm}|}
                \hline
                \textbf{Content-Type} & text/html & text/html & application/xhtml+xml\\
                \hline
                \textbf{doctype} & brak/stary & poprawny & bez znaczenia\\
                \hline
                \textbf{Wynik} & HTML(quirks mode) & HTML 4.01 / HTML 5 & XHTML\\
                \hline
            \end{tabular}
        \end{center}
    \end{table}
    W przypadku trybu html nie zauważy np. zamknięcia elementu typu <strong/>.
\end{document}