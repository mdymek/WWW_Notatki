\documentclass[../main.tex]{subfiles}

\begin{document}

    \subsection{HTTP - HyperText Markup Language.}

    \textbf{XHTML} jest znacznie bardziej restrykcyjny (zgodny z XML).
    \begin{itemize}
        \item Znaczniki i atrybuty w muszą być pisane małymi literami
        \item Wszystkie elementy muszą być poprawnie zakończone (np. <br/>)
        \item Wszystkie wartości atrybutów muszą być w cudzysłowiach lub apostraofach
        \item Każdy atrybut musi mieć wartość (niedozwolone jest np. <div hidden>; musimy pisać <div hidden="hidden">)
        \item Bardziej surowe zasady zagnieżdżania (np. <a> nie może być zagnieżdżone w <a>, <form> nie może zawierać <form> i inne).
        \item Znaki $< >$ \& ' " muszą być reprezentowane przez encje (\&lt; \&gt; \&amp; \&apos; \&quot;) nawet jeśli są w wartościach atrybutów.
        \item W XHTML aby identyfikować fragmenty dokumentu opisane przez znaczniki <a> <form> <img> <map> (i kilka innych) nie należy używać atrybutu name (zamiast tego id)
        \item Dokument musi zaczynać się od deklaracji <!DOCTYPE html PUBLIC "-//W3C//DTD XHTML 1.0 Transitional//EN"
        "http://www.w3.org/TR/xhtml1/DTD/xhtml1-transitional.dtd">
        i musi zawierać elementy <html>, <head>, <title> i <body>.
    \end{itemize}

    \textbf{HTML 5}.
    \begin{itemize}
        \item Nie wymaga aby kod HTML był dobrze uformowanym XMLem.
        \item Dużo wymogów dotyczących poprawności dokumentów i ich dokładniejsze sprawdzanie.
        \item Możliwość zastosowania narzędzi pochodzących z xml takich jak XPath czy XSLT.
        \item Wsparcie przestrzeni nazw co umożliwia użycie np. MathML czy SVG.
        \item Zintegrowane wsparcie dla multimediow niewymagające użycia wtyczek
        \item Brak wersji (jak Strict, Frames, Transitional).
        \item Dobrze tolerowany przez stare przeglądarki.
        \item Nowe elementy, np.: article, aside, audio, footer, header.
        \item Nowe typy elementów formularzy: dates and times, email, url, search, number, range, tel, color.
        \item Nowe atrybuty: charset (dla elementu meta), async (dla elementu script)
        \item Globalne atrybuty (które mogą być zasotsowane do dowolnego elementu): id, tabindex, hidden, data-* (custom data attributes)
        \item Usunięto niektóre zdeprecjonowane elementy: acronym, applet, basefont, big, center, dir, font, frame, frameset, isindex, noframes, strike, tt
        \item Zmiana znaczenia/użycia niektórych istniejących atrybutów i elementów (np <i> i <small> - zmiana znaczenia na bardziej semantyczne)
    \end{itemize}

    \textbf{Nowe API}
    \begin{itemize}
        \item \textbf{GeoLocation API} - dostęp do lokalizacji użytkownika
        \item \textbf{Web Storage API} - wygodniejsze składowanie infomracji po stronie klienta
        \item \textbf{Web Socket API} - full-duplex komunikacja z serwerem.
        \item \textbf{Web Worker API} - umożliwia uruachamianie zdaań w Javascripcie w innym wątku niż interfejs przeglądarki.
        \item \textbf{Drag \& Drop API} - wsparcie dla "drag and drop"
    \end{itemize}

    \textbf{Content type}
    \begin{itemize}
        \item Dla zawartości w standarcie HTML: text/html
        \item Dla zawartości w standarcie XHTML: application/xhtml+xml (ewentualnie application/xml lub text/xml)
    \end{itemize}

    \textbf{Interpretacja przez przeglądarkę.}
    W zależności od Content type oraz <!doctype> przeglądarka interpretuje zawartość w rózny sposób.

    \begin{table}[H]
        \begin{center}
            \begin{tabular}{|p{4cm}|p{4cm}|p{4cm}|p{4cm}|}
                \hline
                \textbf{Content-Type} & text/html & text/html & application/xhtml+xml\\
                \hline
                \textbf{doctype} & brak/stary & poprawny & bez znaczenia\\
                \hline
                \textbf{Wynik} & HTML(quirks mode) & HTML 4.01 / HTML 5 & XHTML\\
                \hline
            \end{tabular}
        \end{center}
    \end{table}
    W przypadku trybu html nie zauważy np. zamknięcia elementu typu <strong />.



\end{document}