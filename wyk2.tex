\documentclass[../main.tex]{subfiles}

\begin{document}
    \begin{table}[H]
        \begin{center}
            \begin{tabular}{|p{8cm}|p{8cm}|}
                \hline
                \multicolumn{2}{|c|}{\textbf{Internet i WWW to nie to samo}}\\
                \hline
                \textbf{Internet} & \textbf{WWW}\\
                \hline
                \hline
                The Internet is a \textbf{global system of interconnected computer networks that interchange data} by
                packet switching using the standardized Internet Protocol Suite (TCP/IP).
                &
                The World Wide Web (WWW, or simply Web) is an \textbf{information space} in which the items of interest,
                referred to as \textbf{resources, are identified by global identifiers} called Uniform Resource
                Identifiers (URI)\\
                \hline
            \end{tabular}
        \end{center}
    \end{table}

    \textbf{Jak działa WWW?}
    \begin{enumerate}
        \item Użytkownik wpisuje w polu adresu przeglądarki adres (URL)
        \item Przeglądarka analizuje wpisany adres sprawdzając protokół (HTTP) oraz nazwę hosta.
        \item Przeglądarka łączy się z serwerem DNS i uzyskuje adres IP hosta.
        \item Przeglądarka nawiązuje połączenie TCP/IP serwerem WWW działającym pod danym adresem.
        \item Przeglądarka wysyła w ramach protokołu HTTP żądanie (request) pobrania zasobu: GET /home.html HTTP/1.1
        \item Serwer lokalizuje żądany zasób statyczny lub generuje go dynamicznie, a następnie zwraca w postaci odpowiedzi (response) HTTP zawierającej treść strony w postaci HTML.
        \item Przeglądarka wyświetla zawartość HTML w postaci graficznej.
    \end{enumerate}

    Powstało w roku \textbf{1990}.

    \subsection{URI}
    \textbf{URI (Uniform Resource Identifier)} - ciąg znaków \textbf{jednoznacznie identyfikaujący} dowolny abstrakcyjny lub fizyczny zasób.
    \begin{itemize}
        \item \textbf{Uniform} - pozwala w \textbf{jednorodny} sposób odwoływać się do różnych rodzajów zasobów niezależnie od sposobu dostępu do nich. Definiuje ogólną postać identyfikatora pozwalając na rozszerzanie i uzupenianie definicji dla różnych typów zasobów. (http, ftp, isbn)
        \item \textbf{Resource} -dowolny abstrakcyjny lub fizyczny \textbf{zasób}, który może być identyfikowany przez URI.
        \item \textbf{Identifier} - identyfikator pozwalający na \textbf{odróżnienie} jednego zasobu od drugiego. Nie musi zawierać informacji o sposobie uzyskania dostępu do tego zasobu.
    \end{itemize}
    URI definiuje jedynie ogólne ramy schematów nazewnictwa (\textbf{naming schemes}), podczas gdy szczegóły składni są
    definiowane w specyfikacjach poszczególnych schematów.

    \textbf{Każde} URI może być \textbf{zaklasyfikowane} jako \textbf{lokalizator URL}, \textbf{nazwa URN} lub
    \textbf{oba} na raz.

    \begin{itemize}
        \item \textbf{URL} to URI określające również \textbf{lokalizację} zasobu (np. adres sieciowy).
        \item \textbf{URN}, to identyfikator nadający zasobowi \textbf{"nazwę"} - \textbf{identyfikator}, który pozostaje \textbf{globalnie unikalny} i trwały nawet
        jeśli sam zasób przestanie istnieć lub stanie się niedostępny.
    \end{itemize}
    \textbf{Sam schemat nie przesądza o tym, czy URI jest URN}. Np. przestrzenie nazw w XML są URN-ami, chociaz na ogół
    korzystają ze schematu http.

    \subsubsection{Składnia URI}
    \begin{lcverbatim}
        scheme:[//[user[:password]@]host[:port]][/path][?query][#fragment]
    \end{lcverbatim}
    \begin{itemize}
        \item \textbf{scheme} - schemat, np. http, https, mailto, ftp, file. Odpowiada często \textbf{protokołowi dostępu} do zasobu, ale nie zawsze tak być musi. Np. dla file nie ma konkretnego protokołu.
        \item \textbf{authority} - składa się z:
        \begin{itemize}
            \item opcjonalnej sekcji autentykacji: username:password zakończonej znakiem (@)
            \item \textbf{host} - w postaci nazwy hosta, adresu IPv4 lub IPv6,
            \item opcjonalnego portu oddzielonego od hosta dwukropkiem (:).
        \end{itemize}
        \item \textbf{path} - ścieżka do zasobu. Jeśli w URI występowała sekcja authority, to ścieżka musi zaczynać się pojedynczym ukośnikiem. Nie może się zaczynać podwójnym ukośnikiem.
        \item \textbf{query} - opcjonalne zapytanie jest oddzielone od wcześniejszych części pytajnikiem (?). Na ogół składa się z \textbf{zestawu par} <atrybut>=<wartość> rozdzielonych ampersandem (\&) lub średnikiem (;).
        \item \textbf{fragment} - opcjonalny fragment oddzielony od wcześniejszych części haszem (\#). Zawiera \textbf{identyfikator zasobu podrzędnego}. Może to być np. sekcja w dokumencie. W przypadku dokumentu HTML jest to na ogół identyfikator pewnego elementu w dokumencie, do krórego przeglądarka przewija dokument podczas wyświetlania.
        Uwaga: Fragment jest obsługiwany tylko w przeglądarce i \textbf{nie jest wysyłany do serwera}.
    \end{itemize}

    \subsection{Protokół HTTP}

    \begin{itemize}
        \item \textbf{Hypertext Transfer Protocol} - protokół do \textbf{wymiany danych w sieci} Internet. HTTP jest \textbf{niesymetrycznym} protokołem request-response klient-serwer.
        \item Protokół HTTP jest \textbf{bezstanowy}, co znaczy, że serwer nie jest w stanie bez dodatkowej informacji, zawartej w żądaniu, stwierdzić że poszczególne żądania należą do danej konwersacji między klientem a serwerem. \textbf{Serwer nie wie, co działo się w poprzednich żądaniach.}
        \item HTTP pozwala klientowi i serwerowi na \textbf{negocjację} typów danych i reprezentacji przesyłanych informacji.
    \end{itemize}

    \begin{table}[H]
        \begin{center}
            \begin{tabular}{|p{8cm}|p{8cm}|}
                \hline
                \textbf{HTTP request} & \textbf{HTTP response}\\
                \hline
                \hline
                \begin{itemize}
                    \item linię \textbf{żądania} (request line),
                    \item \textbf{nagłówki} (headers) - przesyłane w postaci \textbf{par} nazwa:wartość, oddzielone przecinkami.
                    \item \textbf{(pusta linia)}
                    \item opcjonalnie \textbf{treść} żądania (request body)
                \end{itemize}
                &
                \begin{itemize}
                    \item linia \textbf{statusu} (status line),
                    \item \textbf{nagłówki} (headers),
                    \item \textbf{(pusta linia)}
                    \item opcjonalnie \textbf{treść} odpowiedzi (response body)
                \end{itemize}\\
                \hline
            \end{tabular}
        \end{center}
    \end{table}

    \begin{table}[H]
        \begin{center}
            \begin{tabular}{|p{3cm}|p{13.5cm}|}
                \hline
                \multicolumn{2}{|c|}{\textbf{HTTP METHODS}}\\
                \hline
                \textbf{Method} & \textbf{Description}\\
                \hline
                \hline
                \textbf{GET} & The GET method is used to \textbf{retrieve information} from the given server using a
                given URI. Requests using GET should only retrieve data and should have no other effect on the data.\\
                \hline
                \textbf{HEAD} & Same as GET, but it transfers the \textbf{status line and the header section only}.
                (Eg. check if current cached version is correct by checking the "last-modified" header).\\
                \hline
                \textbf{POST} & A POST request is used to \textbf{send data} to the server.\\
                \hline
                \textbf{PUT} & \textbf{Replaces} all the current representations of the target resource with the
                uploaded content.\\
                \hline
                \textbf{DELETE} & \textbf{Removes} all the current representations of the target resource given by URI.\\
                \hline
                \textbf{CONNECT} & \textbf{Establishes a tunnel} to the server identified by a given URI.\\
                \hline
                \textbf{OPTIONS} & \textbf{Describe the communication options} for the target resource.\\
                \hline
                \textbf{TRACE} & Ask the server to return a \textbf{diagnostic trace} of the actions it takes.\\
                \hline
                \textbf{PATCH} & \textbf{Modify part} of the content (instead of replace as in PUT).\\
                \hline
            \end{tabular}
        \end{center}
    \end{table}

    \begin{table}[H]
        \begin{center}
            \begin{tabular}{|p{4cm}|p{6cm}|p{6cm}|}
                \hline
                \multicolumn{3}{|c|}{\textbf{PORÓWNANIE GET I POST}}\\
                \hline
                &\textbf{GET} & \textbf{POST}\\
                \hline
                \hline
                \textbf{Przycisk Wstecz/ Odświeżenie strony} & Niegroźne &
                Dane zostaną powtórnie przesłane. (przeglądarka powinna ostrzec użytkownika)\\
                \hline
                \textbf{Zakładki} & Można dodać do zakładek & Nie można dodac do zakładek\\
                \hline
                \textbf{Cachowanie} & Można cachować & Nie można cachować\\
                \hline
                \textbf{Encoding type} & application/x-www-form-urlencoded &
                application/x-www-form-urlencoded lub multipart/form-data. Dla danych binarnych używamy multipart
                encoding\\
                \hline
                \textbf{Historia} & Parametry pozostają w historii przeglądarki & Parametry nie są przechowywane w historii
                przeglądarki\\
                \hline
                \textbf{Ograniczenia na długość przekazywanych danych} & Metoda GET dodaje parametry do URL-a, a wiele
                przeglądartek ogranicza długość URLa (np. do 2048 znaków) & Brak ograniczeń\\
                \hline
                \textbf{Ograniczenia na przesyłane dane} & Tylko znaki ASCII & Brak ograniczeń (również binarne)\\
                \hline
                \textbf{Bezpieczeństwo} & GET mniej bezpieczny niż POST bo dane są przekazywan jako część URLa.
                Nigdy nie należy przesyłać haseł i innych wrażliwych danych przy pomocy GET &
                POST jest \textbf{tylko odrobinę bezpieczniejszy} niż GET, ponieważ dane nie są przechowywane w historii
                przeglądarki ani w logach serwera.\\
                \hline
            \end{tabular}
        \end{center}
    \end{table}


    \begin{table}[H]
        \begin{center}
            \begin{tabular}{|p{3cm}|p{4cm}|p{8cm}|}
                \hline
                \multicolumn{3}{|c|}{\textbf{STATUSY ODPOWIEDZI}}\\
                \hline
                \hline
                \textbf{1xx} & \textbf{Informational} & \textbf{Request received, server is continuing the process.}\\
                \hline
                100 & Continue & The server received the request and is in the process of giving the response.\\
                \hline
                \hline
                \textbf{2xx} & \textbf{Success}
                & \textbf{The request was successfully received, understood, accepted and serviced.}\\
                \hline
                200 & OK & The request is fulfilled.\\
                \hline
                \hline
                \textbf{3xx} &
                \textbf{Redirection} &
                \textbf{Further action must be taken in order to complete the request.}\\
                \hline
                301 & Move Permanently & The resource requested for has been permanently moved to a new location. The
                URL of the new location is given in the response header called Location. The client should issue a new
                request to the new location. Application should update all references to this new location.\\
                \hline
                302 & Found \& Redirect (or Move Temporarily) & Same as 301, but the new location is temporarily in
                nature. The client should issue a new request, but applications need not update the references.\\
                \hline
                304 & Not Modified& In response to the If-Modified-Since conditional GET request, the server notifies
                that the resource requested has not been modified.\\
                \hline
                \hline
                \textbf{4xx} &\textbf{Client Error} &
                \textbf{The request contains bad syntax or cannot be understood.}\\
                \hline
                400 & Bad Request & Server could not interpret or understand the request, probably syntax error in the
                request message.\\
                \hline
                401 & Authentication Required & The requested resource is protected, and require client’s credential
                (username/password). The client should re-submit the request with his credential (username/password).\\
                \hline
                403 & Forbidden & Server refuses to supply the resource, regardless of identity of client.\\
                \hline
                404 & Not Found & The requested resource cannot be found in the server.\\
                \hline
                405 & Method Not Allowed& The request method used, e.g., POST, PUT, DELETE, is a valid method. However,
                the server does not allow that method for the resource requested.\\
                \hline
                408 & Request Timeout &\\
                \hline
                414 & Request URI too Large &\\
                \hline
                \hline
                \textbf{5xx} &
                \textbf{Server Error} &
                \textbf{The server failed to fulfill an apparently valid request.} \\
                \hline
                500 & Internal Server Error & Server is confused, often caused by an error in the server-side program
                responding to the request.\\
                \hline
                501 & Method Not Implemented & The request method used is invalid (could be caused by a typing error,
                e.g., "GET" misspell as "Get").\\
                \hline
                502 & Bad Gateway & Proxy or Gateway indicates that it receives a bad response from the upstream server.\\
                \hline
                503 & Service Unavailable & Server cannot response due to overloading or maintenance. The client can
                try again later.\\
                \hline
                504 & Gateway Timeout & Proxy or Gateway indicates that it receives a timeout from an upstream server.\\
                \hline
            \end{tabular}
        \end{center}
    \end{table}

\end{document}