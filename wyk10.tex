\documentclass[../main.tex]{subfiles}

\begin{document}

    \begin{itemize}
        \item Protokół HTTP jest bezstanowy (chociaż sam serwer oczywiście jakiś stan posiada).
        \item Każde żądanie przychodzące z serwera jest przetwarzane niezależnie od ewentualnych wcześniejszych żądań
        \item W wielu sytuacjach konieczne jest powiązanie kolejnych żądań przychodzących od tego samego klienta
        \item Ciąg takich żądań przychodzących od jednego klienta, to sesja
        \item Grupowanie żadań przychodzących od jednego klienta (jedna sesja)musi byc realizowane przez programistę (przy ewentualnym wsparciu wykorzystywanych frameworków i bibliotek)
    \end{itemize}

    \begin{theorem}
        Sesja jest nowa jeśli została utworzona po stronie serwera,
        ale klient do niej jeszcze nie przystąpił
    \end{theorem}

    \textbf{Identyfikator sesji}
    \begin{itemize}
        \item Z sesją wiążemy pewien identyfikator generowany przez serwer (identyfikator sesji
        \item Identyfikator sesji jest generowany przez serwer przy pierwszym żądaniu przychodzącym od klienta
        \item Wszystkie następne żądania przychodzące od tego samego klienta zawierają ten identyfikator
    \end{itemize}

    \textbf{Techniki przekazywania identyfikatora sesji}
    \begin{itemize}
        \item Przepisywanie URL (URL rewriting)
        \item Ukryte pola w formularzach HTML
        \item Ciasteczka (cookies)
    \end{itemize}

    \textbf{Przepisywanie URL (URL Rewriting)}
    \begin{itemize}
        \item Identyfikator sesji jest dołączany do URL-a każdego żądania.
        \item Serwer na zwracanej stronie dołącza do wszystkich URL-i dodatkowy parametr z identyfikatorem sesji.
        \item Wywołanie przez klienta takiego URL-a pozwala serwerowi na pobranie identyfikatora sesji z przekazanego parametru
    \end{itemize}


    \begin{table}[H]
        \begin{center}
            \begin{tabular}{p{8cm} | p{8cm}}
                \textbf{Zalety} & \textbf{Wady}\\
                \hline
                \begin{enumerate}
                    \item Działa z każdą przeglądarką niezależnie od ustawień użytkownika
                \end{enumerate}
                &
                \begin{enumerate}
                    \item Trzeba przepisać każdy używany link
                    \item W przypadku potrzeby przekazania większej liczby informacji przez parametry URL-a możemy osiągnąć limit długości URL-a.
                \end{enumerate}\\
            \end{tabular}
        \end{center}
    \end{table}

    \textbf{Ukryte pola formularza (hidden fields)}
    \begin{itemize}
        \item Identyfikator sesji (i ewentualne inne dane) są przekazywane z serwera do przeglądarki jako wartości ukrytych pól formularza
        \item W przeglądarce po wykonaniu submit, zawartość tych poł jest dołączana automatycznie do wartości pozostałych pól i, jako parametry metody POST lub
        GET, są przekazywane z powrotem na serwer.
    \end{itemize}


    \textbf{Http Session}
    \begin{itemize}
        \item W serwletach mamy do dyspozycji interfejs HttpSession
        \item Kontener serwletów używa tego iterfejsu do utworzenia trwałes sesji trwającej określany przedział czasu i rozciągającej się wiele requestów od tego samego użytkownika
        \item Konkrena implementacja zależy od serwera i jego konfiguracji i może opierac się np. na ciasteczkach lub przepisywaniu URLi
        \item Interfejs HttpSession umożliwia
        \begin{itemize}
            \item Podgląd i modyfikację infomracji o sesji, takich jak identyfikator sesji, moment utworzenia, czas ostatniego dostępu, itp.
            \item Dodawanie do sesji obiektów przechowujących informację dostępną
            podczas następnych wywołań serwletu dla tej samej sesji
        \end{itemize}
        \item Interfejs HttpSession
    \end{itemize}

    \textbf{Cookies}
    \begin{itemize}
        \item W kodzie serwlet pobieramy z request ciasteczka (jeśli są)oraz tworzymy nowe i dodajemy do response
        \item Przeglądarka automatycznie do zwrotnego `request`u dołącza ciasteczka
        przesłane z serwera
        \item HTTP Cookie niewielki zestaw informacji przekazywany z serwera do klienta
        \item Klient może takie ciasteczko zachować i odesłać z powrotem z następnym
        zapytaniem do serwera
        \item \textbf{Session cookies} - są usuwane przez przeglądarkę kiedy przeglądarka jest zamykana. nie mają
        ustawionej dyrektywy MaxAge ani Expires)
        \item \textbf{Permanent cookies} - dezaktywuja się w określonym moomencie (dyrektywa Expires) lub po
        określonym czasie (dyrektywa MaxAge). Mogą trwać pomiędzy kolejnymi
        uruchomieniami klienta.
    \end{itemize}

    \textbf{Inne ważniejsze dyrektywy}
    \begin{itemize}
        \item \textbf{Secure} - ciasteczko może być wysłane na serwer jedynie przy użyciu bezpiecznego
        protokołu HTTPS
        \item \textbf{HttpOnly} - ciasteczko może być wysłane na serwer jedynie przez przeglądarkę. Nie jest
        możliwe dołączenie ciasteczka z pzoiomu kodu w Javascript.
        \item \textbf{Domain} - fomena (nazwa hosta) do której ciasteczko może być wysłane.
        Dopuszczalne są także poddomeny. Jeśli Domain nie jest utawione, to można odesłać ciasteczko jedynie do
        domeny z której przyszło (wykluczając podomeny)
    \end{itemize}

    \textbf{Servlet filtr}
    \begin{itemize}
        \item Filtr jest obiektem, który jest wywoływany przed obsłużeniem przez serwlet żadania klienta oraz po jego obsłużeniu.
        \item Może odczytywać i modyfkować zawartość żądania oraz odpowiedzi
        \item Zastosowania filtrów
        \begin{itemize}
            \item zapisywanie do logów
            \item kompresja
            \item szyfrowanie i deszyfrowanie
            \item autoryzacja użytkownika
            \item walidacja dostępu do zasobów
        \end{itemize}
    \end{itemize}

    \subsection{Autentykacja użytkownika}
    Metody
    \begin{enumerate}
        \item HTTP basic authentication
        \begin{itemize}
            \item Mechanizm wbudowany w protokół HTTP
            \item Schematy: Basic (base64), Bearer(OAuth 2.0), Digest (md5, sha itp)
            \item Basic authentication scheme:
            \begin{itemize}
                \item Identyfikator użytkonika wraz z hasłem przesyłane są w kodowaniu base64, które jest odwracalne
                \item Powininen (w właściwie musi) być używany z protokołem HTTPS aby uchronić hasło przed dostępem osób trzecich
                \item Wsparcie dla tej motody jest wbudowane zarówno w kliencie jak i na serwerze
                \item Brak możliwości wylogowania użytkonika
            \end{itemize}
        \end{itemize}
        \item Cookies
        \begin{itemize}
            \item Po przesłaniu danych autoryzacyjnych serwer generuje identyfikator sesji autoryzowanej.
            \item Jest on przechowywany na serwerze oraz przekazany do klienta
            \item Klient przekazuje go przy każdym żądaniu dostęu do zasobów
            \item Serwer sprawdza, czy identyfikator jest poprawny i aktywny
            \item Wylogowanie polega na dezaktywacji lub usunięcia identyfikatora z serwera
            \item Brak możliwości wyklogowania użytkonika, ale można określić czas ważności tokena
        \end{itemize}
        \item Tokens
        \begin{itemize}
            \item Najczęściej używane sa JSON Web Tokens (JWT)
            \item Token nie jest przechowywany na serwerze, a tylko u klienta (Local storage, Cookies, itp)
            \item Klient wraz zżadaniem dostęu do zasobu wysyła token w nagłówku
            Authorization: Bearer <token>.
            \item Serwer weryfikuje poprawność tokenu przy pomocy swojego prywatnego klucza
            \item Struktura tokena:
            \begin{itemize}
                \item Nagłówek (header) - określa typ tokenu u rodzaj lagorytmu haszującego.
                \item Zawartość (payload) - Zawiera stwierdzenia najczęściej na temat
                tożsamości użytkonika (sub - subject)
                \item Podpis (signature) - podpis kryptograficzny nagłówka i zawartości.
            \end{itemize}
        \end{itemize}
    \end{enumerate}
\end{document}

