\documentclass[../main.tex]{subfiles}

\begin{document}

    \begin{itemize}
        \item Protokół HTTP jest \textbf{bezstanowy} (chociaż sam serwer oczywiście jakiś stan posiada).
        \item \textbf{Każde żądanie} przychodzące jest \textbf{przetwarzane niezależnie} od ewentualnych wcześniejszych żądań
        \item W wielu sytuacjach konieczne jest \textbf{powiązanie kolejnych żądań} przychodzących od tego samego klienta
        \item Ciąg takich żądań przychodzących od jednego klienta, to \textbf{sesja}
        \item \textbf{Grupowanie żadań} przychodzących od jednego klienta musi byc \textbf{realizowane przez programistę}
    \end{itemize}

    \begin{theorem}
        Sesja jest \textbf{nowa} jeśli {została utworzona} po stronie serwera,
        ale klient do niej jeszcze \textbf{nie przystąpił}.
    \end{theorem}

    \subsection{Identyfikator sesji}
    \begin{itemize}
        \item Z sesją wiążemy pewien \textbf{identyfikator generowany} przez serwer (identyfikator sesji)
        \item Identyfikator sesji jest generowany przez serwer \textbf{przy pierwszym żądaniu} przychodzącym od klienta
        \item \textbf{Wszystkie następne żądania} przychodzące od tego samego klienta \textbf{zawierają} ten identyfikator
    \end{itemize}

    \subsection{Techniki przekazywania identyfikatora sesji}
    \begin{itemize}
        \item \textbf{Przepisywanie URL} (URL rewriting)
        \item \textbf{Ukryte pola} w formularzach HTML
        \item \textbf{Ciasteczka} (cookies)
    \end{itemize}

    \subsubsection{Przepisywanie URL (URL Rewriting)}
    \begin{itemize}
        \item Identyfikator sesji jest \textbf{dołączany do URL}-a każdego żądania.
        \item Serwer na zwracanej stronie dołącza do wszystkich URL-i \textbf{dodatkowy parametr} z identyfikatorem sesji.
        \item Wywołanie przez klienta takiego URL-a pozwala serwerowi na \textbf{pobranie identyfikatora sesji} z przekazanego parametru
    \end{itemize}

    \begin{table}[H]
        \begin{center}
            \begin{tabular}{p{8cm} | p{8cm}}
                \textbf{Zalety} & \textbf{Wady}\\
                \hline
                \begin{enumerate}
                    \item \textbf{Działa z każdą przeglądarką} niezależnie od ustawień użytkownika
                \end{enumerate}
                &
                \begin{enumerate}
                    \item Trzeba \textbf{przepisać każdy} używany \textbf{link}
                    \item W przypadku potrzeby przekazania większej liczby informacji przez parametry URL-a \textbf{możemy osiągnąć limit długości URL}-a.
                \end{enumerate}\\
            \end{tabular}
        \end{center}
    \end{table}

    \subsubsection{Ukryte pola formularza (hidden fields)}
    \begin{itemize}
        \item Identyfikator sesji (i ewentualne inne dane) są \textbf{przekazywane} z serwera do przeglądarki jako \textbf{wartości ukrytych pól formularza}
        \item W przeglądarce po wykonaniu \textbf{submit}, zawartość tych pól jest \textbf{dołączana automatycznie} do wartości pozostałych pól i, jako parametry metody POST lub
        GET, są przekazywane z powrotem na serwer.
    \end{itemize}

    \subsubsection{Cookies}
    \begin{itemize}
        \item W kodzie serwlet pobieramy z \textbf{request ciasteczka} (jeśli są) oraz \textbf{tworzymy nowe} i dodajemy do \textbf{response}
        \item Przeglądarka \textbf{automatycznie} do zwrotnego request'u \textbf{dołącza} ciasteczka
        przesłane z serwera
        \item HTTP Cookie to \textbf{niewielki zestaw informacji} przekazywany z serwera do klienta
        \item Klient może takie ciasteczko \textbf{zachować i odesłać} z powrotem z następnym
        zapytaniem do serwera
        \item \textbf{Session cookies} - są \textbf{usuwane przez przeglądarkę} kiedy przeglądarka jest \textbf{zamykana}. Nie mają
        ustawionej dyrektywy MaxAge ani Expires.
        \item \textbf{Permanent cookies} - \textbf{dezaktywuja się} w określonym \textbf{moomencie} (dyrektywa Expires) lub po
        określonym \textbf{czasie} (dyrektywa MaxAge). Mogą trwać pomiędzy kolejnymi uruchomieniami klienta.\\

        \item \textbf{Inne ważniejsze dyrektywy}:
        \begin{itemize}
            \item \textbf{Secure} - ciasteczko może być wysłane na serwer jedynie przy użyciu \textbf{bezpiecznego
            protokołu} HTTPS
            \item \textbf{HttpOnly} - ciasteczko może być wysłane na serwer \textbf{jedynie przez przeglądarkę}. Nie jest
            możliwe dołączenie ciasteczka z poziomu kodu w Javascript.
            \item \textbf{Domain} - domena (nazwa hosta) do której ciasteczko może być wysłane.
            Dopuszczalne są także poddomeny. Jeśli Domain \textbf{nie jest utawione}, to można odesłać ciasteczko jedynie do
            \textbf{domeny z której przyszło} (wykluczając podomeny).
        \end{itemize}
    \end{itemize}

    \subsection{Http Session}
    \begin{itemize}
        \item W \textbf{serwletach} mamy do dyspozycji \textbf{interfejs HttpSession}.
        \item Kontener serwletów używa tego iterfejsu do utworzenia \textbf{trwałej sesji} trwającej \textbf{określany przedział czasu} i rozciągającej się wiele requestów od tego samego użytkownika.
        \item Konkretna implementacja zależy od serwera i jego konfiguracji i może opierac się np. na \textbf{ciasteczkach} lub \textbf{przepisywaniu URLi}.
        \item Interfejs HttpSession umożliwia:
        \begin{itemize}
            \item \textbf{Podgląd i modyfikację informacji} o sesji, takich jak identyfikator sesji, moment utworzenia, czas ostatniego dostępu, itp.
            \item \textbf{Dodawanie} do sesji \textbf{obiektów} przechowujących informację dostępną
            podczas następnych wywołań serwletu dla tej samej sesji
            \item Wykorzystuje nagłowek \textbf{Authorization}.
        \end{itemize}
    \end{itemize}



    \subsection{Filtr Servlet}
    \begin{itemize}
        \item Filtr jest obiektem, który jest \textbf{wywoływany przed obsłużeniem} przez serwlet \textbf{żądania} klienta oraz \textbf{po} jego obsłużeniu.
        \item Może \textbf{odczytywać i modyfkować} zawartość \textbf{żądania} oraz \textbf{odpowiedzi}.
        \item \textbf{Zastosowania} filtrów"
        \begin{itemize}
            \item zapisywanie do \textbf{logów}
            \item \textbf{kompresja}
            \item \textbf{szyfrowanie} i deszyfrowanie
            \item \textbf{autoryzacja} użytkownika
            \item \textbf{walidacja dostępu} do zasobów
        \end{itemize}
    \end{itemize}

    \subsection{Autentykacja użytkownika}
    \subsubsection{HTTP basic authentication}
    \begin{itemize}
        \item Mechanizm \textbf{wbudowany} w protokół HTTP
        \item Schematy: Basic (base64), Bearer(OAuth 2.0), Digest (md5, sha itp)
        \item \textbf{Basic} authentication scheme:
        \begin{itemize}
            \item Identyfikator użytkownika wraz z hasłem przesyłane są w kodowaniu \textbf{base64}, które jest \textbf{odwracalne}
            \item \textbf{Powininen} (właściwie musi) być używany z protokołem \textbf{HTTPS} aby uchronić hasło przed dostępem osób trzecich
            \item Wsparcie dla tej motody jest wbudowane zarówno w kliencie jak i na serwerze
            \item \textbf{Brak możliwości wylogowania} użytkowika
        \end{itemize}
    \end{itemize}

    \subsubsection{Cookies}
    \begin{itemize}
        \item Po przesłaniu danych autoryzacyjnych serwer generuje \textbf{identyfikator sesji autoryzowanej}.
        \item Jest on \textbf{przechowywany na serwerze} oraz \textbf{przekazany do klienta}
        \item Klient \textbf{przekazuje} go\textbf{przy każdym żądaniu} dostęu do zasobów
        \item Serwer \textbf{sprawdza}, czy identyfikator jest \textbf{poprawny} i \textbf{aktywny}
        \item \textbf{Wylogowanie} polega na \textbf{dezaktywacji} lub \textbf{usunięcia} identyfikatora z serwera
    \end{itemize}

    \subsubsection{Tokeny}
    \begin{itemize}
        \item Najczęściej używane sa JSON Web Tokens (\textbf{JWT})
        \item \textbf{Brak możliwości wylogowania użytkownika}, ale można określić \textbf{czas ważności tokena}
        \item Token \textbf{nie jest przechowywany na serwerze}, a tylko u klienta (Local storage, Cookies, itp)
        \item \textbf{Klient} wraz z żądaniem dostępu do zasobu \textbf{wysyła} token w \textbf{nagłówku Authorization: Bearer} <token>.
        \item \textbf{Serwer weryfikuje poprawność} tokenu przy pomocy swojego \textbf{prywatnego klucza}
        \item Struktura tokena:
        \begin{itemize}
            \item \textbf{Nagłówek} (header) - określa \textbf{typ} tokenu i \textbf{rodzaj algorytmu haszującego}.
            \item \textbf{Zawartość} (payload) - Zawiera stwierdzenia najczęściej na temat
            \textbf{tożsamości użytkownika} (sub - subject)
            \item \textbf{Podpis} (signature) - \textbf{podpis kryptograficzny} nagłówka i zawartości.
        \end{itemize}
    \end{itemize}
\end{document}