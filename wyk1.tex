\documentclass[../main.tex]{subfiles}

\begin{document}

    \begin{itemize}
        \item \textbf{Frontend}
        \begin{itemize}
            \item HTML - podstawowy język znaczników dla opisu stron www
            \item CSS (Cascading Style Sheets) - Kaskadowe arkusze stylów - Język do definiowania takich atrybutów, które wpływają na wygląd strony jak: kolor, czcionka, układ (wyrównanie, centrowanie, marginesy), itp.
            \item JavaScript - Jeden z najpopularniejszych języków ostatnich lat. De-facto standard dla programowania po stronie klienta. Umożliwia tworzenie interaktywnych stron.
            \item jQuery - jedna z najczęściej stosowanych bibliotek JavaScript do manipulowania elementami strony WWW.
            \item Pozostałe - to bardziej rozbudowane biblioteki lub frameworki do budowy aplikacji WWW.
        \end{itemize}
        \item \textbf{Backend}
        \begin{itemize}
            \item Java - jeden z najpopularnieszych i najbardziej uniwersalnych języków programowania.
            \item Serwlety - niewielkie aplikacje (najczęsciej w Javie) rozszerzające mozliewości swerwera WWW. Odpowiadają na żadania, które serwer otrzymuje od klienta (np. przeglądarki). Najczęściej generują dynamiczne strony WWW lub zwracają dane np. pobrane z bazy danych.
            \item Java Server Pages (JSP) - technologia generowania stron WWW. W specyficznym języku znaczników będącym mieszanką HTML-a i Javy tworzone są szablony stron, które są następnie tłumaczone na serwlety w Javie. Przykład pliku JSP
            \item Spring MVC - framework będący częścią Spring Framework pozwalający ułatwiający tworzenie aplikacji webowych bazujących na serwletach w modelu Model-View-Controller
            \item Node.js - wieloplatformmowe środowisko uruchomieniowe (runtime) JavaScript działające po stronie serwera. Umożliwia między innymi łatwe tworzenie serwerów WWW w Javascripcie.
            \item Django - framework w Pythonie do szybiego tworzenia aplikacji internetowych bazujących na wzorcu model-template-view
            \item Ruby on Rails - framework w języku Ruby do tworzenia aplikacji internetowych
            \item ASP.NET - framework od Microsoftu do tworzenia dynamicznych aplikacji internetowych przy użyciu m.in. języków C\# i VB.NET
            \item Go - język stworzony w firmie Google. Doskonale nadaje się do tworzenia aplikacji internetowych (po stronie serwera).
            \item Swift - język stworzony przez Apple dla iOS itp. Na jego bazie istinieją frameworki umożliające tworzenie aplikacji webowch
        \end{itemize}
    \end{itemize}


    \begin{table}[H]
        \begin{center}
            \begin{tabular}{|p{8cm}|p{8cm}|}
                \hline
                \multicolumn{2}{|c|}{\textbf{Czym jest aplikacja www?}}\\
                \hline
                \textbf{Klasyczne serwisy WWW} & \textbf{Single Page Applications}\\
                \hline
                \hline
                \begin{itemize}
                    \item Ładowane są całe strony.
                    \item Strony są generowane przez skrypty CGI, serwlety Java, itp.
                    \item Zmiana zawartości strony wymaga załadowania całej strony
                    \item Cała logika aplikacji (lub jej większa część) działa po stronie serwera (backend)
                \end{itemize}
                &
                \begin{itemize}
                    \item Cała strona(y) jest ładowana jednorazowo na początku
                    \item Cała logika aplikacji działa po stronie klienta (przeglądarki). Zaimplementowana w JavasScript.
                    \item Różne fragmenty aplikacjji są wyświetlane i/lub tworzone dynamicznie w czasie działania aplikacji
                    \item Potrzebne dane (w postaci JSON, XML, itp) są pobierane z serwera przy pomocy wywołań AJAX
                \end{itemize}\\
                \hline
            \end{tabular}
        \end{center}
    \end{table}

\end{document}