\documentclass[../main.tex]{subfiles}

\begin{document}
    \begin{itemize}
        \item \textbf{Środowisko uruchomieniowe Javascript} oparte na silniku Google Chrome V8.
        \item Stworzony w \textbf{2009} roku przez Ryana Dahla w celu \textbf{wyeliminowania problemów ze skalowalnością} tradycyjnych serwerów HTTP.
        \item \textbf{Open source}: licencja MIT.
        \item Wykorzystywany przede wszystkim do tworzenia \textbf{serwerów} HTTP, ale jest też używany do tworzenia aplikacji desktopowych.
    \end{itemize}

    \subsection{Architektura}
    \begin{itemize}
        \item Jeden wątek z \textbf{pętlą zdarzeń} (event loop),
        \item \textbf{Nieblokujące I/O} bazujące na zdarzeniach i wywołaniach zwrotnych (\textbf{callback}),
        \item Bazuje na bibliotece libuv (napisanej w języku C).
    \end{itemize}

    \begin{table}[H]
        \begin{center}
            \begin{tabular}{ p{9cm} | p{8cm} }
                \textbf{Zalety} & \textbf{Wady}\\
                \hline
                \begin{itemize}
                    \item \textbf{Duża szybkość}, potrafi obsłużyć znacznie więcej żądań niż tradycyjny serwer (jak np. Apache),
                    \item \textbf{Brak narzutów} związanych z tworzeniem i utrzymywaniem \textbf{wątków},
                    \item Wysoka \textbf{skalowalność} związana z użyciem nieblokującego API i pojedynczej pętli zadarzeń,
                    \item \textbf{Rozbudowany ekosystem} narzędzi i bibliotek,
                    \item Możliwość pisania frontendu i backendu w tym samym języku,
                    \item Jest częścią środowisk tworzenia aplikacji webowych:
                    \begin{itemize}
                        \item \textbf{MEAN} - MongoDB, Express, Angular, Node,
                        \item \textbf{MERN} - MongoDB, Express, React, Node.
                    \end{itemize}
                \end{itemize}
                &
                \begin{itemize}
                    \item Niezbyt dobrze sprawdza się w zastosowaniach wymagających \textbf{dużej mocy obliczeniowej} (CPU intensive),
                    \item Stosunkowo \textbf{duża zmienność} całego środowiska i pakietów,
                    \item \textbf{Callback hell} – problem z utrzymaniem \textbf{przejrzystości} kodu z związku z potrzebą stosowania dużej liczby wywołań zwrotnych.
                \end{itemize}\\
            \end{tabular}
        \end{center}
    \end{table}

    \subsection{NPM - Node Package Manager}
    \begin{itemize}
        \item Podstawowe \textbf{narzędzie} do \textbf{zarządzania pakietami} w Node.js.
        \item Umożliwia instalację i aktualizację pakietów z uwzględnieniem \textbf{zależności} między nimi.
        \item Pakiety można instalować \textbf{globalnie} lub \textbf{lokalnie}.
        \item \textbf{Plik package.json}
        \begin{itemize}
            \item Zawiera ważne \textbf{informacje} na temat tworzonego pakietu/aplikacji.
            \item Zawiera \textbf{listę pakietów} od których zależy pakiet/aplikacja wraz z \textbf{numerami wersji}.
            \item Pozwalana na \textbf{automatyczne pobranie} wszystkich \textbf{niezbędnych zależności}.
        \end{itemize}
    \end{itemize}
\end{document}
