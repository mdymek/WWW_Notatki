\documentclass[../main.tex]{subfiles}

\begin{document}

    \begin{itemize}
        \item Biblioteka w JavaScript do tworzenia \textbf{interfejsu użytkownika}
        \item Stworzona przez Facebooka
        \item Jedna z najpopularniejszych bibliotek JavaScript
        \item Nie jest pełnym frameworkiem (jak np. Angular)\\

        \item \textbf{Jednokierunkowy przepływ danych} (single-way data flow): "properties flow down; actions flow up"
        \item \textbf{Virtual DOM} (wirtualny obiektowy model dokumentu) - React tworzy w pamięci swój własny DOM w którym monitoruje
        wszystkie zmiany i uaktualnia widok w przeglądarce tylko o faktyczne zmiany.
        \item \textbf{JSX} (JavaScript Syntax eXtension) - pozwala na \textbf{wprowadzanie składni HTML}-o podobnej bezpośrednio w kodzie JavaScript
        \item React Native - biblioteki dla IOS-a, Androida
        \item \textbf{Props vs. State}
        \begin{itemize}
            \item \textbf{props} - dane przekazane do komponentu (jak parametry wywołania funkcji)
            \item \textbf{state} - stan wewnętrzny komponentu. zmieniany \textbf{asynchronicznie} przez setState().
        \end{itemize}
    \end{itemize}
\end{document}